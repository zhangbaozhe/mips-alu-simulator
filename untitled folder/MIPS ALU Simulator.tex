% Options for packages loaded elsewhere
\PassOptionsToPackage{unicode}{hyperref}
\PassOptionsToPackage{hyphens}{url}
%
\documentclass[
]{article}
\usepackage{lmodern}
\usepackage{amssymb,amsmath}
\usepackage{ifxetex,ifluatex}
\ifnum 0\ifxetex 1\fi\ifluatex 1\fi=0 % if pdftex
  \usepackage[T1]{fontenc}
  \usepackage[utf8]{inputenc}
  \usepackage{textcomp} % provide euro and other symbols
\else % if luatex or xetex
  \usepackage{unicode-math}
  \defaultfontfeatures{Scale=MatchLowercase}
  \defaultfontfeatures[\rmfamily]{Ligatures=TeX,Scale=1}
\fi
% Use upquote if available, for straight quotes in verbatim environments
\IfFileExists{upquote.sty}{\usepackage{upquote}}{}
\IfFileExists{microtype.sty}{% use microtype if available
  \usepackage[]{microtype}
  \UseMicrotypeSet[protrusion]{basicmath} % disable protrusion for tt fonts
}{}
\makeatletter
\@ifundefined{KOMAClassName}{% if non-KOMA class
  \IfFileExists{parskip.sty}{%
    \usepackage{parskip}
  }{% else
    \setlength{\parindent}{0pt}
    \setlength{\parskip}{6pt plus 2pt minus 1pt}}
}{% if KOMA class
  \KOMAoptions{parskip=half}}
\makeatother
\usepackage{xcolor}
\IfFileExists{xurl.sty}{\usepackage{xurl}}{} % add URL line breaks if available
\IfFileExists{bookmark.sty}{\usepackage{bookmark}}{\usepackage{hyperref}}
\hypersetup{
  hidelinks,
  pdfcreator={LaTeX via pandoc}}
\urlstyle{same} % disable monospaced font for URLs
\usepackage{color}
\usepackage{fancyvrb}
\newcommand{\VerbBar}{|}
\newcommand{\VERB}{\Verb[commandchars=\\\{\}]}
\DefineVerbatimEnvironment{Highlighting}{Verbatim}{commandchars=\\\{\}}
% Add ',fontsize=\small' for more characters per line
\newenvironment{Shaded}{}{}
\newcommand{\AlertTok}[1]{\textcolor[rgb]{1.00,0.00,0.00}{\textbf{#1}}}
\newcommand{\AnnotationTok}[1]{\textcolor[rgb]{0.38,0.63,0.69}{\textbf{\textit{#1}}}}
\newcommand{\AttributeTok}[1]{\textcolor[rgb]{0.49,0.56,0.16}{#1}}
\newcommand{\BaseNTok}[1]{\textcolor[rgb]{0.25,0.63,0.44}{#1}}
\newcommand{\BuiltInTok}[1]{#1}
\newcommand{\CharTok}[1]{\textcolor[rgb]{0.25,0.44,0.63}{#1}}
\newcommand{\CommentTok}[1]{\textcolor[rgb]{0.38,0.63,0.69}{\textit{#1}}}
\newcommand{\CommentVarTok}[1]{\textcolor[rgb]{0.38,0.63,0.69}{\textbf{\textit{#1}}}}
\newcommand{\ConstantTok}[1]{\textcolor[rgb]{0.53,0.00,0.00}{#1}}
\newcommand{\ControlFlowTok}[1]{\textcolor[rgb]{0.00,0.44,0.13}{\textbf{#1}}}
\newcommand{\DataTypeTok}[1]{\textcolor[rgb]{0.56,0.13,0.00}{#1}}
\newcommand{\DecValTok}[1]{\textcolor[rgb]{0.25,0.63,0.44}{#1}}
\newcommand{\DocumentationTok}[1]{\textcolor[rgb]{0.73,0.13,0.13}{\textit{#1}}}
\newcommand{\ErrorTok}[1]{\textcolor[rgb]{1.00,0.00,0.00}{\textbf{#1}}}
\newcommand{\ExtensionTok}[1]{#1}
\newcommand{\FloatTok}[1]{\textcolor[rgb]{0.25,0.63,0.44}{#1}}
\newcommand{\FunctionTok}[1]{\textcolor[rgb]{0.02,0.16,0.49}{#1}}
\newcommand{\ImportTok}[1]{#1}
\newcommand{\InformationTok}[1]{\textcolor[rgb]{0.38,0.63,0.69}{\textbf{\textit{#1}}}}
\newcommand{\KeywordTok}[1]{\textcolor[rgb]{0.00,0.44,0.13}{\textbf{#1}}}
\newcommand{\NormalTok}[1]{#1}
\newcommand{\OperatorTok}[1]{\textcolor[rgb]{0.40,0.40,0.40}{#1}}
\newcommand{\OtherTok}[1]{\textcolor[rgb]{0.00,0.44,0.13}{#1}}
\newcommand{\PreprocessorTok}[1]{\textcolor[rgb]{0.74,0.48,0.00}{#1}}
\newcommand{\RegionMarkerTok}[1]{#1}
\newcommand{\SpecialCharTok}[1]{\textcolor[rgb]{0.25,0.44,0.63}{#1}}
\newcommand{\SpecialStringTok}[1]{\textcolor[rgb]{0.73,0.40,0.53}{#1}}
\newcommand{\StringTok}[1]{\textcolor[rgb]{0.25,0.44,0.63}{#1}}
\newcommand{\VariableTok}[1]{\textcolor[rgb]{0.10,0.09,0.49}{#1}}
\newcommand{\VerbatimStringTok}[1]{\textcolor[rgb]{0.25,0.44,0.63}{#1}}
\newcommand{\WarningTok}[1]{\textcolor[rgb]{0.38,0.63,0.69}{\textbf{\textit{#1}}}}
\usepackage{longtable,booktabs}
% Correct order of tables after \paragraph or \subparagraph
\usepackage{etoolbox}
\makeatletter
\patchcmd\longtable{\par}{\if@noskipsec\mbox{}\fi\par}{}{}
\makeatother
% Allow footnotes in longtable head/foot
\IfFileExists{footnotehyper.sty}{\usepackage{footnotehyper}}{\usepackage{footnote}}
\makesavenoteenv{longtable}
\setlength{\emergencystretch}{3em} % prevent overfull lines
\providecommand{\tightlist}{%
  \setlength{\itemsep}{0pt}\setlength{\parskip}{0pt}}
\setcounter{secnumdepth}{-\maxdimen} % remove section numbering
\ifluatex
  \usepackage{selnolig}  % disable illegal ligatures
\fi
\usepackage{geometry}
\usepackage{fancyhdr}
\geometry{left=2.54cm, right=2.54cm, top=2.54cm, bottom=2.54cm}
\pagestyle{fancy}

\author{ZHANG Baozhe}
\date{\today}

\begin{document}

\hypertarget{header-n0}{%
\section{\texorpdfstring{MIPS ALU Simulator
}{MIPS ALU Simulator }}\label{header-n0}}

\hypertarget{header-n2}{%
\subsection{Project Overview}\label{header-n2}}

This project is a simple MIPS ALU simulator which supports following
instructions:

\begin{verbatim}
- add, addi, addu, addiu
- sub, subu
- and, andi, nor, or, ori, xor, xori
- beq, bne, slt, slti, sltiu, sltu
- lw, sw
- sll, sllv, srl, srlv, sra, srav
\end{verbatim}

Actually, the true ALU in MIPS architecture only achieves: \texttt{and},
\texttt{or}, \texttt{add}, \texttt{subtract},
\texttt{set\ on\ less\ that}, \texttt{nor} by receiving specific opcodes
and contents from two registers (or extended numbers, etc.) and
outputing a 32-bit result and a 3-bit flag. And thus, some functions of
the above instructions cannot be achived by the only ALU (e.g.,
\texttt{lw}).

\hypertarget{header-n13}{%
\subsection{Projects Details}\label{header-n13}}

This ALU is defined by its behaviors (not by its structure) in Verilog.
For the specific file organization and project building instructions,
they can be refered in the README file. Basically, to run the project only type 
\begin{verbatim}
> make
\end{verbatim}
in the terminal.

\hypertarget{header-n17}{%
\subsubsection{How the simulation works}\label{header-n17}}

There are three steps for the simulaton to work: parsing the binary
instruction, reading the input data, and outputing the results. The
details of the three steps are elaborated as follows:

\begin{itemize}
\item
  Parsing:

\begin{Shaded}
\begin{Highlighting}[]
\KeywordTok{always}\NormalTok{ @(ins, regA, regB) }\KeywordTok{begin}
\NormalTok{    opcode = ins[}\DecValTok{31}\NormalTok{:}\DecValTok{26}\NormalTok{];}
\NormalTok{    funct  = ins[}\DecValTok{5}\NormalTok{:}\DecValTok{0}\NormalTok{];}
\NormalTok{    shamt  = ins[}\DecValTok{10}\NormalTok{:}\DecValTok{6}\NormalTok{];}
\NormalTok{    ...}
    \KeywordTok{case}\NormalTok{ (opcode)}
\NormalTok{        ...}
    \KeywordTok{endcase}
\KeywordTok{end}
\end{Highlighting}
\end{Shaded}
\end{itemize}

\begin{itemize}
\item
  Reading data: the instance of \texttt{mips\_alu} module reads the
  instruction, regA, and regB data from three test files:
  \texttt{INS\_DATA}, \texttt{REGA\_DATA}, and \texttt{REGB\_DATA},
  respectively.
\item
  Output: the result of the specific instruction will be assigned to the
  \texttt{result} and \texttt{flag}
\end{itemize}

\hypertarget{header-n53}{%
\subsubsection{Test bench}\label{header-n53}}

There are 135 instructions are tested in the test bench. There are a few
notes in this simulation:

\begin{enumerate}
\def\labelenumi{\arabic{enumi}.}
\item
  Zero detection is simultaneous performed;
\item
  For \texttt{beq} and \texttt{bne}, the negative flag will be set to 1
  if the result is to branch;
\item
  For \texttt{slt} and related instructions, the negative flag will be
  set to 1 if rs \textless{} rt;
\item
  For \texttt{lw} and \texttt{sw}, since there are no memory simulation
  in this project, there are the two instructions only add the numbers
  in rs and the offsets.
\end{enumerate}

Below are the part of the test results (some of the result may not be
resonable, e.g., the inputs of \texttt{lw} or \texttt{sw}):
\newpage
\hypertarget{header-n54}{%
\paragraph{Arithmetic \& logic instructions}\label{header-n54}}
\begin{longtable}[]{@{}llllll@{}}
\toprule
INS(b) & REGA(h) & REGB(h) & RESULT(h) & FLAG(b) & INS
NAME\tabularnewline
\midrule
\endhead
00000000000000010000000000100000 & 00000001 & 00000002 & 00000003 & 000
& ADD\tabularnewline
00000000001000000000000000100000 & 7fffffff & 00000001 & 80000000 & 001
& ADD\tabularnewline
00000000001000000000000000100000 & 80000000 & 80000000 & 00000000 & 101
& ADD\tabularnewline
00000000000000010000000000100001 & ffffffff & 00000001 & 00000000 & 100
& ADDU\tabularnewline
00100000000000011111111111111111 & 80000000 & ffffffff (imm) & 7fffffff
& 001 & ADDI\tabularnewline
00100100000000011111111111111111 & 80000000 & ffffffff (imm) & 7fffffff
& 000 & ADDIU\tabularnewline
00000000000000010000000000100100 & 00000001 & ffffffff & 00000001 & 000
& AND\tabularnewline
00110000000000011111111111111111 & ffffffff & 0000ffff (imm) & 0000ffff
& 000 & ANDI\tabularnewline
00000000001000000000000000100110 & 80000000 & ffffffff & 7fffffff & 000
& XOR\tabularnewline
00000000000000010000000110000000 & xxxxxxxx & 7fffffff & ffffffc0 & 000
& SLL\tabularnewline
00000000000000010000000011000010 & xxxxxxxx & ffffffff & 1fffffff & 000
& SRL\tabularnewline
00000000001000000000000000000110 & 7fffffff & ffffffff & 00000000 & 100
& SRLV\tabularnewline
00000000000000010000011111000011 & xxxxxxxx & 80000000 & ffffffff & 000
& SRA\tabularnewline
00000000001000000000000000000111 & 7fffffff & ffffffff & 00000000 & 100
& SRAV\tabularnewline
00000000001000000000000000100010 & 80000000 & ffffffff & 7fffffff & 001
& SUB\tabularnewline
00000000001000000000000000100011 & 7fffffff & ffffffff & 80000000 & 000
& SUBU\tabularnewline
\bottomrule
\end{longtable}

Note that it is required for the immediate numbers of \texttt{andi},
\texttt{ori}, and \texttt{xori} to be zero-extended.

\hypertarget{header-n312}{%
\paragraph{Branch \& slt instructions}\label{header-n312}}

\begin{longtable}[]{@{}llllll@{}}
\toprule
INS(b) & REGA(h) & REGB(h) & RESULT(h) & FLAG(b) & INS
NAME\tabularnewline
\midrule
\endhead
0001000000100000xxxxxxxxxxxxxxxx & 8000ffff & 80000000 & xxxxxxxx & 000
& BEQ\tabularnewline
0001010000000001xxxxxxxxxxxxxxxx & 00000001 & 00000000 & xxxxxxxx & 100
& BNE\tabularnewline
00000000000000010000000000101010 & 00000001 & ffffffff & 00000000 & 100
& SLT\tabularnewline
00101000000000011111111111111111 & 80000000 & ffffffff (imm) & 00000001
& 010 & SLTI\tabularnewline
\bottomrule
\end{longtable}

\hypertarget{header-n172}{%
\paragraph{Load \& store instructions}\label{header-n172}}

\begin{longtable}[]{@{}llllll@{}}
\toprule
INS(b) & REGA(h) & REGB(h) & RESULT(h) & FLAG(b) & INS
NAME\tabularnewline
\midrule
\endhead
10001100000000011111111111111000 & 0000f001 & 00000009 & 0000f00a & 000
& LW\tabularnewline
10101100000000010000000000001001 & 0000f001 & 00000009 & 0000f00a & 000
& SW\tabularnewline
\bottomrule
\end{longtable}

\end{document}
